\documentclass{beamer}
\usetheme{metropolis}           % Use metropolis theme
\title{\LaTeX \  und Marxismus}
\usepackage{breqn}
\date{\today}
\author{Kimi Müller}
% \institute{Centre for Modern Beamer Themes}
\begin{document}
  \maketitle
  \section{WZF?!}
  \begin{frame}{Zwei Themen?}
    \begin{itemize}
        \item<1-> \textbf{Problem:} Ich möchte gerne über Marxismus reden, habe aber viel Aufwand in die Präsentation gesteckt.
        \item<2-> \textbf{Lösung:} Ich referiere kurz über \LaTeX und danach über Marxismus.
        \item<3-> \textbf{Grundlage:} Thema ist frei wählbar
    \end{itemize}
  \end{frame}
  \begin{frame}
    \includegraphics[width=\textwidth]{Modern_Problems_Require_Modern_Solutions.jpg}
  \end{frame}
  \begin{frame}{Agenda}
\begin{itemize}
    \item<2-> Disclaimer
    \item<3-> Was ist \LaTeX?
    \item<4-> Warum \LaTeX?
    \item<5-> Was ist Kommunismus?
    \item<6-> Mehrwerttheorie
    \item<7-> Stummer Zwang der materiellen Verhältnisse
    \item<8-> Reservearmee des Kapitals
\end{itemize}
  \end{frame}
  \begin{frame}{Disclaimer}
\textbf{Ich bin kein Experte in \LaTeX, Kommunismus oder Präsentationen. Ich liebe die Freiheitlich-Demokratische Rechtsordnung, diese Präsentation dient nur zu theoretischen und Unterhaltungszwecken.}
\end{frame}
  \section{Was ist \LaTeX?}
  \begin{frame}{WYSIWYG}
    \includegraphics[width=\textwidth]{types_of_editors_2x.png}
  \end{frame}
  \begin{frame}{Echtes Beispiel dieser Präsentation}
  \end{frame}
  \section{Warum \LaTeX?}
  \begin{frame}
\begin{dmath}
  Q(\lambda,\hat{\lambda}) = -\frac{1}{2} P{(O \mid \lambda )} \sum_s \sum_m \sum_t \gamma_m^{(s)} (t) \left( n \log(2 \pi ) + \log \left| C_m^{(s)} \right| + \left( \mathbf{o}_t - \hat{\mu}_m^{(s)} \right) ^T C_m^{(s)-1} \left(\mathbf{o}_t - \hat{\mu}_m^{(s)}\right) \right)
\end{dmath}
\end{frame}
\begin{frame}{Dasselbe in Word}
\href{https://i.makeagif.com/media/5-28-2022/dXcfrH.gif}{\includegraphics[width=\textwidth]{FormelnWord.png}}
\end{frame}
\begin{frame}
\href{Link mit einsteigerfreundlichen Informationen}{https://www.overleaf.com/}
\end{frame}

\begin{frame}{Eigenschaften von \LaTeX}
    \begin{itemize}
        \item<2-> Beitrag von Vielen für Viele
        \item<3-> Große Community
        \item<4-> Komplexe Anwendung, aber simples Konzept
        \item<5-> Funktioniert besser als man denkt
        \item<6-> Belächelt von Menschen, die es nicht verstehen
    \end{itemize}

        $\Rightarrow$ auf den ersten Blick unattraktiv, aber unterschätzt.
\end{frame}

\begin{frame}{Eigenschaften von \LaTeX\ und Kommunismus}
    \begin{itemize}
        \item Beitrag von Vielen für Viele
        \item Große Community
        \item Komplexe Anwendung, aber simples Konzept
        \item Funktioniert besser als man denkt
        \item Belächelt von Menschen, die es nicht verstehen
    \end{itemize}
        $\Rightarrow$ auf den ersten Blick unattraktiv, aber unterschätzt.
\end{frame}
\section{Kommunismus}
\begin{frame}{Was ist Kommunismus?}
Wovon ich rede, wenn ich Kommunismus sage
\end{frame}
\begin{frame}
\begin{itemize}
\item In dieser Präsentation: ein überspannendes Framework, um kapitalismusskeptische Ansichten zusammenzufassen
\item Losgelöst von einzelnen Philosophen
\item Viele viele Details, die ich euch erspare.
\item Auszug: (Antideutsche vs. Antiimperialisten, Stalininisten vs Trotskisten, Histomat/Diamat, Ideologiekritik, Warenfetisch, subaltern)
\item 200 Jahre Theorien in diesem Feld läppert sich
\end{itemize}
\end{frame}
\begin{frame}{Mehrwerttheorie - Stellt euch vor...}
\includegraphics[width=\textwidth]{naeherinnen.jpg}
\end{frame}
\begin{frame}{Mehrwerttheorie}
\begin{itemize}
    \item \textbf{Mehrwert}: Die Arbeiter produzieren mehr Wert, als sie in Form von Löhnen zurückerhalten. Die Differenz nennt sich Mehrwert.
    \item \textbf{Akkumulation}: Der vom Arbeiter erschaffene Mehrwert wird vom Kapitalisten reinvestiert, um mehr Kapital zu akkumulieren und somit den Produktionsprozess zu erweitern.
\end{itemize}
\end{frame}
\section{Wie motiviert man Arbeiter zur Arbeit?}
\begin{frame}{Zwei Ansätze, die ineinander greifen}
\begin{itemize}
 \item<1-> Reservearmee des Kapitals
 \item<2-> Stummer Zwang der materiellen Verhältnisse
\end{itemize}
\end{frame}
\begin{frame}{Der stumme Zwang der materiellen Verhältnisse}
\begin{itemize}
 \item das kapitalistische System zwingt die Menschen durch ihre wirtschaftlichen Bedingungen zur Anpassung.
 \item ohne direkte Gewalt oder Gesetze
 \item Arbeiter müssen Arbeitskraft verkaufen, weil sie keine eigenen Produktionsmittel besitzen
 \item Wahl zwischen Verhungern und Arbeiten  $\Rightarrow$ Arbeiten
\end{itemize}
\end{frame}

\begin{frame}{Reservearmee des Kapitals}
\begin{itemize}
    \item Menschen, die arbeitslos sind oder nur zeitweise Arbeit finden
    \item Verstärkt den Druck auf die Arbeiter
    \item Mechanismus:
    \begin{itemize}
        \item Viele Arbeitslose
        \item $\Rightarrow$ Größere Auswahl an Arbeitskräften.
        \item $\Rightarrow$ Arbeiter wissen, dass sie leicht ersetzt werden können.
        \item $\Rightarrow$ Löhne bleiben niedrig.
    \end{itemize}
    \item Durch stetige Produktivitätssteigerung werden immer mehr Menschen arbeitslos. Der Kreislauf wiederholt sich.
\end{itemize}

\end{frame}
\begin{frame}{Panoptikum}
    \includegraphics[width=\textwidth]{Panoptikum}
\end{frame}


\end{document}
